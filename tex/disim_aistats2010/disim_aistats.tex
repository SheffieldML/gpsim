\documentclass[accepted]{article}
\usepackage{aistats2e}
\usepackage{graphicx}
\usepackage{booktabs}
\usepackage{natbib}

% If your paper is accepted, change the options for the package aistats2e as follows
%\usepackage[accepted]{aistats2e}
% This option will print headings for the title of your paper and headings for the
% authors names, plus a copyright note at the end of the first column of the first page.

\begin{document}

% If your paper is accepted and the title of your paper is very long, the style will print
% as headings an error message. Use the following command to supply a shorter title of your
% paper so that it can be used as headings.
%\runningtitle{I use this title instead because the last one was very long}

% If your paper is accepted and the number of authors is large, the style will print
% as headings an error message. Use the following command to supply a shorter version of the authors
% names so that they can be used as headings (for example, use only the surnames)
%\runningauthor{Surname 1, Surname 2, Surname 3, ...., Surname n}

\twocolumn[

\aistatstitle{A model-based method for transcription factor target
  identification from short gene expression time series data}

\aistatsauthor{Antti Honkela$^1$ \And Neil D. Lawrence$^2$ \And Magnus Rattray$^2$}

\aistatsaddress{ 
  $^1$ Dept. of Information and Computer Science,
  Aalto University School of Science and Technology,
  Finland \\
  $^2$ School of Computer Science, University of Manchester, UK } ]

\section{Introduction}

Based on an upcoming journal paper~\citep{Honkela2010PNAS}, we present
a computational method for identifying potential targets of a
transcription factor (TF) using wild-type gene expression time series
data. For each putative target gene we fit a simple differential
equation model of transcriptional regulation and the model likelihood
serves as a score to rank targets~\citep{Lawrence2007,Gao2008}.

The expression profile of the TF is modeled as a sample from a
Gaussian process (GP) prior distribution that is integrated out using a
non-parametric Bayesian procedure. We use a linear differential 
equation model of TF protein translation and target gene transcription~\citep{Gao2008} which, 
although highly simplified, allows for both TF protein decay and target gene mRNA decay. This
is a parsimonious model with relatively few parameters which can be applied to short
time series data sets without noticeable over-fitting. The GP marginal likelihood is exactly tractable
for this model and we therefore avoid fitting functional degrees of freedom by maximum likelihood.

We assess our method using genome-wide Chromatin Immunoprecipitation
(ChIP-chip) and loss-of-function mutant expression data for two TFs,
Twist and Mef2, controlling mesoderm development in {\it
  Drosophila}~\citep{Zinzen2009}. Lists of top-ranked genes identified
by our method are significantly enriched for genes close to bound
regions identified in the ChIP-chip data and for genes that are
differentially expressed in loss-of-function mutants. Targets of Twist
display diverse expression profiles and in this case a model-based
approach performs significantly better than scoring based on
correlation with TF expression. Our approach is found to be comparable
or superior to ranking based on mutant differential expression
scores. Also, we show how integrating complementary wild-type spatial
expression data can further improve target ranking performance.

% Transcription factors are key nodes in the gene regulatory
% networks that determine the function and fate of
% cells. An important first step in uncovering a gene regulatory network
% is the identification of target genes regulated by a specific
% transcription factor (TF). A
% common approach to this problem is to experimentally locate physical binding
% of TF proteins to DNA sequence {\em in vivo} using a genome-wide
% chromatin immunoprecipitation (ChIP)
% experiment~\cite{Ren2000,Iyer2001}. However, recent studies suggest
% that many observed binding events are neutral and do not regulate
% transcription, while regulatory binding events often occur at
% enhancers that are not proximal to the target gene that they
% control~\cite{Li2008,MacArthur2009}. Therefore, the task of
% identifying transcriptional
% targets requires the integration of ChIP binding predictions
% with evidence from expression data to help associate binding events
% with target gene regulation. If there is access to expression data from a mutant in which the TF has
% been knocked out or over-expressed, then differential expression of genes between wild-type
% and mutant is indicative of a potential regulatory
% interaction~\cite{Furlong2001,Sandmann2006a}. Available spatial expression data for the
% TF and putative target can also provide support for a hypothesised
% regulatory link. 

% A problem with the above approach is that the creation of mutant
% strains is challenging or impossible for many TFs of interest. Even
% when available, mutants may provide very limited information because of redundancy or due to the
% confounding of signal from indirect regulatory feedback~\cite{Gitter2009}. For these reasons it
% is useful to seek other sources of evidence to complement ChIP binding
% predictions. In this contribution we demonstrate how a dynamical model of wild-type transcriptional
% regulation can be used for genome-wide scoring of putative target genes. All that is required to apply
% our method is wild-type time series data collected over a period
% where TF activity is changing. Our approach allows for complementary
% evidence from expression data to be integrated with ChIP binding data
% for a specific TF without carrying out TF perturbations. 

% To score putative targets we use the data likelihood under a simple
% cascaded differential equation model of transcriptional regulation. The regulation model
% we apply is ``open'' in the sense that we do not explicitly model regulation of the TF
% itself. To deal with this technical issue we use a recently developed
% non-parametric probabilistic inference methodology to
% effectively deal with open differential equation
% systems~\cite{Gao2008}. We model the TF concentration as a function
% drawn from a Gaussian process prior distribution~\cite{Rasmussen2006,Lawrence2010}.
% This functional prior can either be placed
% on the TF mRNA, for TFs primarily under transcriptional regulation,
% or the TF protein, for TFs activated post-translationally. In the
% application considered here the TFs are transcriptionally regulated
% and we take the former approach. We use Bayesian marginalisation (also
% known as Bayesian model averaging) to
% integrate out these functional degrees of freedom. This greatly reduces the
% number of parameters required to model the data, making a
% likelihood-based approach feasible even for short time series.

% There are many existing approaches to inferring gene regulatory networks from
% time series expression data, including dynamic Bayesian networks,
% information theoretic approaches and differential equation approaches
% (reviewed in \cite{Bansal2007a}). These methods typically require many
% more data from a greater diversity of experimental conditions than are
% available from the short unperturbed wild-type time series that we
% consider. Indeed, most real gene expression time course data are short
% relative to the simulated data used to assess computational methods
% for network inference~\cite{Ernst2005}. However, our goal is more limited in scope since
% we are primarily interested in providing additional support for hypothesised
% targets of a specific TF. Again, most approaches to this problem are
% designed for data containing large numbers of diverse conditions, as
% exemplified by the DREAM 2 target identification
% challenge 1~\cite{Stolovitzky2007}. Others
% have addressed this target identification problem using time series
% data with a regulation model~\cite{Barenco2006a,Gatta2008}. However,
% these approaches either require a known target set for training~\cite{Barenco2006a} or
% they require measured TF protein data~\cite{Gatta2008}. In addition to
% these differences in the assumed prior knowledge and available data,
% it is also difficult to validate other approaches on the same data used in these
% studies as they only carried out validation of selected targets
% that they identified, rather than using unbiased genome-wide
% experimental validation. 

% We validate our proposed method by comparing the model-based target
% ranking with published ChIP-chip data for two TFs controlling
% mesoderm development in \emph{Drosophila}. Our method is shown to be comparable
% to, or outperform, the use of knockout mutant data which are available
% for these TFs. We demonstrate that a model-based approach is significantly
% better than a simpler approach using correlation when targets display a diverse set of expression profiles. We further show how integrating complementary wild-type spatial
% {\em in situ} expression data can greatly improve target ranking accuracy. 

\subsubsection*{Acknowledgements}

A.H. was supported by Postdoctoral Researcher's Project No 121179 of
the Academy of Finland.  M.R. and N.D.L acknowledge support from EPSRC
Grant No EP/F005687/1 ``Gaussian Processes for Systems Identification
with Applications in Systems Biology".  This work was supported in
part by the IST Programme of the European Community, under the PASCAL2
Network of Excellence, IST-2007-216886. This publication only reflects
the authors' views.

\bibliographystyle{myabbrvnat}
\bibliography{disim}

\end{document}
