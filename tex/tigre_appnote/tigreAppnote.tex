\documentclass{bioinfo}
\copyrightyear{2010}
\pubyear{2010}

\newcommand{\tigre}{\emph{tigre}}

\begin{document}
\firstpage{1}

\title[tigre]{tigre: Transcription factor Inference through Gaussian process Reconstruction of Expression for Bioconductor}
\author[Honkela \textit{et~al.}]{Antti Honkela\,$^{1,*}$, Pei Gao\,$^{2}$, Jonatan Ropponen\,$^{1}$, Magnus Rattray\,$^{3,*}$ and Neil D.\ Lawrence\,$^3$\footnote{to whom correspondence should be addressed}}
\address{$^{1}$Department of Information and Computer Science, Aalto
  University School of Science and Technology, Helsinki, Finland\\
  $^{2}$Department of of Public Health and Primary Care, University of
  Cambridge, Cambridge, UK\\
  $^{3}$Department of Computer Science, University of Sheffield,
  Sheffield, UK}

\history{Received on XXXXX; revised on XXXXX; accepted on XXXXX}

\editor{Associate Editor: XXXXXXX}

\maketitle

\begin{abstract}

\section{Summary:}
\tigre{} is an R/Bioconductor package for inference of transcription
factor activity and ranking candidate target genes from gene
expression time series.  The underlying methodology is based on
Gaussian process inference on a differential equation model that
allows using short unevenly sampled time series.  The method has been
designed with efficient parallel implementation in mind, and the
package supports several alternative approaches for parallel operation
on commodity hardware.

\section{Availability:}
The package is included in Bioconductor release 2.6 for R 2.11.

\section{Contact:} \href{antti.honkela@tkk.fi}{antti.honkela@tkk.fi}
\end{abstract}

\section{Introduction}

Understanding genome function through reverse engineering of gene
regulatory relationships from experimental data is one of the key
challenges in current biology~\citep{ENCODE2007,Bickel2009e}.  One
popular technique is to use gene expression time series to infer these
relationships.  Unfortunately most real world expression time series
are short~\citep{Ernst2005} and contain insufficient information for
any realistic reconstruction of the gene regulatory
network~\citep{Smet2010}.

Recognising this, the \tigre{} package aims at answering a much
simpler question: given time series expression data where a
transcription factor (TF) is changing its activity, is a gene
plausibly regulated by that TF.  As a result, it provides a ranking of
tested target genes according to their likelihood of being targets of
the TF.

The underlying methodology was presented by~\citet{Honkela2010PNAS},
who showed that it can yield remarkably accurate predictions from very
limited data, often attaining more accurate results based on the
simple wild time series expression data than could be obtained using
TF knock-out data.

\begin{methods}
\section{Methods}

The \tigre{} package is an implementation of the Gaussian process
single input motif framework of~\citet{Gao2008} and the related TF
target ranking method of~\citet{Honkela2010PNAS}.  This framework is
based on a linear ordinary differential equation model of TF protein
translation and transcription regulation described by the equations
\begin{align}
  \frac{\mathrm{d}p(t)}{\mathrm{d}t} & = f(t) - \delta
  p(t) \ , \label{eq:translation_ode} \\
  \frac{\mathrm{d}m_j(t)}{\mathrm{d}t} & = B_j+S_j p(t)-D_j m_j(t) \ , \label{eq:transcription_ode}
\end{align}
where $p(t)$ is the TF protein and $m_j(t)$ is the $j$th target mRNA
concentration at time $t$. The parameters $B_j$, $S_j$ and $D_j$ are the
baseline transcription rate, sensitivity and decay rate respectively
for the mRNA of the $j$th target (as described by \citet{Barenco2006a}).
The parameter $\delta$ is the decay rate of the TF
protein~\citep{Honkela2010PNAS}.

Placing a Gaussian process prior on $f(t)$\footnote{If the TF protein
  is under significant post-translational regulation,
  Eq.~(\ref{eq:translation_ode}) may be omitted and the prior placed
  directly on $p(t)$.  In this case multiple known targets are needed
  to reliably infer $p(t)$.} leads to a joint Gaussian process over
all continuous-time activity functions.  The parameters of the model
as well as other parameters of the Gaussian process covariance are
optimised by maximising the marginal likelihood.
\end{methods}

\section{Implementation}

The \tigre{} package is tightly integrated into Bioconductor
microarray data analysis framework, especially with the \emph{puma}
package~\citep{Pearson2009}.

Functions are provided for processing data to a format suitable for
the method, including estimation of variances of absolute expression
values for \emph{puma} processed data; fitting the models individually
or in a batch; and plotting the models to assess the fit.  The models
are fitted using scaled conjugate gradient
optimisation~\citep{Moller:scg93}.  Fitted models can be stored very
compactly by just storing their parameters.  They can also be easily
recreated afterwards without rerunning the optimiser used in the
fitting.

\subsection{Parallelisation}

The method implemented by \tigre{} includes no Monte Carlo simulations
and is thus relatively fast, but still takes some time, ranging from
seconds to up to a few minutes per gene depending on the data and the
number of targets in the models.

\tigre{} has been designed for efficient parallelisation.  In the
ranking, each gene is handled completely independently.  This makes
the code trivially parallelisable up to the level of running each gene
in a separate machine.  This linear parallelisation to potentially
several thousands of processes is impossible in more tightly coupled
modelling methods.

The easiest way to run \tigre{} in parallel is to simply split the
task to a number of jobs that can be run independently, possibly by
submitting them as independent jobs to a queuing system.  The package
does not include integration with an MPI environment, because that
matches its requirements poorly.  A number of independent jobs should
also be easier to schedule than a single large MPI job.

An alternative technique for running \tigre{} in parallel is based on
MapReduce~\citep{Dean2008}.  The ranking approach fits this paradigm
perfectly: the mapper fits models to each gene independently and the
reducer forms the final ranking.  We have implemented this approach
using Hadoop and RHIPE.  This approach also allows relatively easy
running of the ranking in a highly parallelised fashion in a cloud
computing setting.

\section{Discussion}

\tigre{} can provide useful results on very short time series.  We
have successfully applied it to data sets with as few as 6 and 7 time
points~\citep{Honkela2010PNAS,Honkela2010MLSP}.

The linearity of the differential equation transcription model greatly
simplifies the algorithm, but it may be too crude assumption for some
situations.  We are working on an method based on a more realistic
model.  However, it seems difficult to turn that framework into
something as convenient as \tigre{}, which nicely captures the
essential degrees of freedom in the transcription regulatory process.

\section*{Acknowledgement}

The authors wish to thank Jennifer Withers for useful comments on the
package.

\paragraph{Funding\textcolon}
A.H. was supported by Postdoctoral Researcher's Project No 121179 of the Academy of Finland.
M.R. and N.D.L acknowledge support from EPSRC Grant No EP/F005687/1 "Gaussian Processes for Systems Identification with Applications in Systems Biology". 
This work was supported in part by the IST Programme of the European Community, under the PASCAL2 Network of Excellence, IST-2007-216886. This publication only reflects the authors' views.

\small

\bibliographystyle{myabbrvnat}
\bibliography{tigre}

\end{document}
